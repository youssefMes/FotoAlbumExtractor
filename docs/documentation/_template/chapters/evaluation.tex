% !TEX root = ../ausarbeitung.tex

\chapter{Evaluation}
Der Evaluationsteil sollte zunächst die Forschungsfragen/anvisierten Beiträge aus der Einleitung aufgreifen und daraus die verwendeten Evaluationsmethoden ableiten. Z.B. \qq{Das Ziel war es eine barrierefreie, effizente Bedienoberfläche zu erstellen, die auf verschiedenen Endgeräten läuft.} Daraus ergeben sich folgende Evaluationsziele und Methoden:
\begin{itemize}
  \item barrierefrei $\rightarrow$ Heuristiken zur Barrierefreiheit nach W3C oder BITV
  \item effizient $\rightarrow$ Messung der Antwortzeit für verschiedene Anfragen
  \item verschiedene Endgeräte $\rightarrow$ Anzeigebeispiele von verschiedenen Endgeräten oder Test durch entsprechendes Tool, das verschiedene Endgeräte simuliert.
\end{itemize}

Es bietet sich an, die Durchführung der Evaluation und ihre Beschreibung parallel zu bearbeiten, denn oft fällt erst beim Schreiben auf, dass die Evaluationsmethode nicht zu den Zielen der Arbeit passt. Die Beschreibung sollte alle Parameter enthalten, die den Versuch ausmachen, d.h. Programmparameter, Testbedingungen, Anzahl Testläufe etc.

\hfil\rule{0.4\textwidth}{0.4pt}

\Blindtext[6]