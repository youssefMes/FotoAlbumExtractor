\documentclass[twoside,12pt,a4paper]{scrreprt}
\usepackage[T1]{fontenc}
\usepackage[utf8]{inputenc}
\usepackage[ngerman]{babel}
\usepackage{babelbib}
\usepackage{parskip}
\usepackage{microtype}
\usepackage{graphicx} % Zum Einbinden von Grafiken
\usepackage[dvipsnames]{xcolor}
\usepackage[colorlinks=true,linkcolor=Black,citecolor=MidnightBlue,urlcolor=MidnightBlue]{hyperref}
\usepackage[all]{hypcap}
\usepackage{pgfplots} \pgfplotsset{compat=1.9}
\usepackage{helvet} % Schönere SansSerif-Schrift
\usepackage{times}  % Schönere Serif-Schrift

\usepackage{blindtext} % sollte am Ende nicht mehr benötigt werden ;)

\pagestyle{headings}

\graphicspath{ {figures/} } % Pfad-Prefix für einzubindende Grafiken. Es sind auch mehrere Pfade möglich, diese müssen jeweils in eigenen {Klammern} stehen.

\setkomafont{disposition}{\normalcolor\bfseries} % überall Serifen verwenden
% oder
%\renewcommand{\familydefault}{\sfdefault} % überall Sans-Serif verwenden

% PDF-Optionen (werden in den Dateieigenschaften angezeigt)
\hypersetup{
pdftitle={Titel der Arbeit},
pdfauthor={Vor- und Nachname},
pdfsubject={Bachelorarbeit Informatik},
pdfpagelayout=TwoColumnRight
}

%%% Eigene Makros
\newcommand{\qq}[1]{\glqq #1\grqq} % \qq{Text in Anführungszeichen}

\begin{document}

%%% Titelseite
\begin{titlepage}
\begin{center}
\LARGE Eberhard Karls Universität Tübingen\\
\large Mathematisch-Naturwissenschaftliche Fakultät \\
Wilhelm-Schickard-Institut für Informatik\\
[3cm]
\huge Bachelorarbeit Informatik\\
[2cm]
\Large\textbf{Titel der Arbeit (der geht wohl in den\\ meisten Fällen über mehr als eine Zeile)}\\
[1.5cm]
\large Vor- und Nachname\\
[0.5cm]
Datum\\
\vfill
\small\textbf{Gutachter}\\[0.3cm]
\large Name Gutachter\\
\footnotesize Wilhelm-Schickard-Institut für Informatik\\Universität Tübingen\\
[1cm]
\small\textbf{Betreuer}\\[0.3cm]
\large Name Betreuer\\
\footnotesize Adresse\\
Universität Tübingen
\end{center}
\end{titlepage}

%%% Titelrückseite: Bibliographische Angaben
\thispagestyle{empty}
\vspace*{\fill}
\textbf{Nachname, Vorname:}\\
\emph{Titel der Arbeit}\\
Bachelorarbeit Informatik\\
Eberhard Karls Universität Tübingen\\
Bearbeitungszeitraum: Anfangs- -- Enddatum
\newpage

%%% Zusammenfassung (Abstract), hier aus externer Datei eingebunden
% !TEX root = ../ausarbeitung.tex

\begin{abstract}
\section*{Zusammenfassung}
Wissenschaftliche Arbeiten fangen normalerweise mit einer kurzen Zusammenfassung an. Deshalb sollte Ihre Arbeit ebenfalls eine solche Zusammenfassung enthalten. Die Zusammenfassung hat einen ähnlichen Inhalt wie die Motivation, nur viel kürzer. Sie soll kurz beschreiben
\begin{itemize}
\item worum es in der Arbeit geht (was war das zu lösende Problem?),
\item welche Methoden zur Problemlösung angewendet wurden,
\item wie das ganze evaluiert wurde,
\item evtl. welches Ergebnis/ Schlussfolgerungen sich daraus ergeben.
\end{itemize}

\hfil\rule{0.4\textwidth}{0.4pt}

Dieses Dokument soll als Ausgangs-Template für Bachelorarbeiten dienen. Gleichzeitig soll es zeigen, wie so ein \qq{fertiges Dokument} aussehen könnte. Um die Seiten gefüllt zu bekommen, wurde Blindtext verwendet.

Die kurzen Beschreibungen zu den Abschnitten (jeweils über dem Querstrich) wurden \cite{alexandrakirsch2016} entnommen. Diese \qq{Hinweise zum Erstellen von Bachelor-/Masterarbeiten im Arbeitsbereich Mensch-Computer-Interaktion und Künstliche Intelligenz} sind aber auch darüber hinaus zu empfehlen.
\end{abstract}
\newpage

%%% Inhaltsverzeichnis
\KOMAoption{toc}{listof,bib} % Abbildungs-/Tabellenverzeichnis, Literaturverzeichnis aufnehmen
\tableofcontents\label{toc}
\cleardoublepage

%%% Hauptteil (mit \input{dateiname} wird die Datei 'dateiname' eingebunden)
\chapter{Einleitung}

% problemstellung
% ziele definieren
% vorhandene daten beschreiben
\section{Problemstellung}
% - Ausschneiden von Fotos aus einem Fotoalbum \\
% - Fotoalbumseite sauber eingescannt \\
% - Bilder mit Rahmen, der entfernt werden soll \\
% - Gesichtserkennung mit Ausschneiden der Gesichter \\
% - Seite eines Fotoalbums besteht aus Hintergrund und eingeklebten Fotos \\
% - Einfaches modulares Programm ohne grafische Oberfläche \\
% - Erstellen von Metriken um die Ergebnisse zu überprüfen

Die hier entwickelte Software soll aus einer eingescannten Fotoalbumseite die darin enthaltenen Fotos extrahieren. Dazu sollte die Seite des Fotoalbums sauber eingescannt sein mit einem möglichst gleichfarbigen Hintergrund und ohne Spiegelungen. Die Fotos können einen Rahmen haben, der von dem Programm entfernt wird.
Sobald die Fotos ausgeschnittenen wurden kann für jedes einzelne Foto innerhalb der Fotoalbumseite eine Gesichtserkennung durchgeführt werden, bei der die erkannten Gesichter markiert und ausgeschnitten werden.
Um das Ergebnis der Software zu überprüfen soll es möglich sein die ausgeschnittenen Fotos mit Ground Truth Bildern auf Ähnlichkeit zu vergleichen. Das Programm soll über die Kommandozeile ausgeführt werden und in einzelne Module aufgeteilt sein, sodass einzelne Elemente jeder Zeit angepasst werden können.
 
\section{Benutzte Technologien}
% genutzte methoden, toolboxen, programmiersprachen
Für die Entwicklung dieser Software wurde als Programmiersprache Python 3.6 verwendet zusammen mit der Bibliothek numpy in der Version 1.14.2. Numpy ermöglicht die schnellere Berechnung von Matrizen und wird für die Bibliothek OpenCV in der Version 3.4.0.12 benötigt. OpenCV bietet eine reihe von Algorithmen für die Bildbearbeitung. Um eine Auswertung anzufertigen wurde aus der Bibliothek SciKit-image 0.13.1 die Methode zur Berechnung der strukturellen Ähnlichkeit (SSIM) verwendet.

\section{Programmaufbau}
Das Programm teilt sich in fünf Hauptkomponenten auf. Die erste ist in der \textit{main.py} zu finden, die das Programm startet und die anderen Komponenten ausführt. Die zweite Komponente kümmert sich um das entfernen des allgemeinen Hintergrunds und ist in der \textit{backgroundremover.py} zu finden. In dieser Komponente werden die eigentlichen Fotos aus dem Fotoalbum grob ausgeschnitten. Danach wird in der nächsten Komponente für jedes Foto der Rahmen entfernt, falls ein Rahmen vorhanden ist. Das entfernen des Rahmens geschieht in der \textit{rectextract.py}. Die letzte Komponente ermöglicht es Gesichter in den ausgeschnittenen Bildern zu erkenne. Dies geschieht in der \textit{facedetection.py}. Möchte man die ausgeschnittenen Bilder mit Beispielbildern vergleichen, so kann man die letzte Komponente in der Datei \textit{compare.py} verwenden. Mit deren Hilfe Metriken aufgestellt werden zum Vergleich der Bilder. In dem Flussdiagramm \ref{fig:flowchart} ist der Ablauf des Programmes nochmals genauer als Diagramm dargestellt.

\begin{figure}[h]
	\centering
	\includegraphics[width=0.45\linewidth]{images/flowchart.png}
	\caption{Flussdiagramm, das den Ablauf des Programmes darstellt. Verzweigungen können mit Hilfe von Parametern bei der Ausführung des Programmes gesteuert werden.}
	\label{fig:flowchart}
\end{figure}

\cleardoublepage

% !TEX root = ../ausarbeitung.tex

\chapter{Stand der Forschung}
Hier zeigen Sie, dass Sie über Ihr Themengebiet gut informiert sind. Sie können entweder den Stand der Forschung dafür heranziehen, um Ihr Thema zu rechtfertigen (\qq{Warum ist es wichtig?}) oder Sie können die Literatur als Grundlage Ihrer Diskussion verwenden (Wie ordnen sich Ihre Beiträge in die Wissenschaftslandschaft allgemein ein?), eine Mischung ist auch möglich.

\hfil\rule{0.4\textwidth}{0.4pt}

Dazu gehören natürlich Referenzen zu anderen Werken. Für deren Verwaltung empfiehlt sich BibTeX, die entsprechende Datei \verb|literatur.bib| (kann natürlich auch anders benannt sein) ist in der Vorlage schon enthalten. Zur Einhaltung der Syntax bietet sich ein Online-Editor wie \cite{BibTexOnlineEditor} an; TeXstudio \cite{texstudio} hält im Menü auch Hilfe beim Erstellen der Bibliographie-Einträge bereit. Für Bücher bietet z.B. auch \url{https://books.google.de/} fertige BibTeX-Einträge an.

Die Einbindung in den Text erfolgt dann mit \verb|\cite{nielsen1994usability}|, wobei der Text in den Klammern durch das in der \verb|.bib|-Datei vergebene Kürzel ersetzt werden müssen. Das Ergebnis ist dann eine Referenz zum (zumindest im Bereich \qq{Usability Engineering}) fast unverzichtbaren gleichnamigen Buch \cite{nielsen1994usability} von Jakob Nielsen.

Achtung: Ihre Ausarbeitung sollte sich (im Gegensatz zu diesem Template) weniger auf Internet-Quellen, als auf Bücher und Paper stützen!
\\

\Blindtext[5]
\cleardoublepage

% !TEX root = ../ausarbeitung.tex

\chapter{Herangehensweise}

Im Hauptteil beschreiben Sie Ihre praktische Arbeit. Code gehört normalerweise nicht in eine Ausarbeitung. Ausnahmen sind Algorithmen, die für Sie wichtig waren (dann in möglichst übersichtlichem Pseudo-Code). Kleine Code-Stücke können auch zur Illustration oder als Beispiel eingebaut werden. Längere Code-Stücke können im Anhang untergebracht werden. Sie sollten jedoch nicht den gesamten Code im Anhang abdrucken. Ihren Code geben Sie bitte dennoch mit ab, am besten auf einer CD.

Der Detailgrad sollte so sein, dass ein Leser die gleiche Arbeit noch einmal nachimplementieren könnte. Insbesondere sollten alle Parameter, von denen die Funktionsweise des Systems abhängt, explizit genannt sein. Bei der Evaluation muss bei jedem Versuch angegeben werden, mit welchen Parametern gearbeitet wird.


\hfil\rule{0.4\textwidth}{0.4pt}

\section{Einführung in \LaTeX}
Wenn Sie dieses Template gefunden haben, werden Sie schon erkannt haben, dass \LaTeX\ nicht ganz unwichtig ist. Falls Sie noch kaum oder keine Erfahrung damit haben, bietet sich der Kurs \qq{Einführung in \LaTeX} von Thorsten Nagel an. Er findet meistens mehrmals im Semester statt, die genauen Termine entnehmen Sie bitte dem Vorlesungsverzeichnis. Das Skript zu diesem Kurs ist online erhältlich \cite{thorstennagel2015} und enthält noch viele weitere Informationen, die in der folgenden Kurzzusammenfassung nicht berücksichtigt werden konnten.

\section{Sinnvolle Programme}
Falls Sie \verb|.tex|-Dateien nicht unbedingt in einem normalen Texteditor schreiben und auf der Kommandozeile kompilieren möchten, bietet sich ein spezieller \LaTeX-Editor wie z.B. TeXstudio \cite{texstudio} an.

\section{Wichtige \LaTeX-Befehle}
\subsection{Textgliederung}
Mit dem Befehl \verb|\section{Überschrift}| lässt sich ein neuer Abschnitt beginnen, der automatisch nummeriert und im \hyperref[toc]{Inhaltsverzeichnis} eingetragen wird.
Gleiches gilt für einen Unterabschnitt, der mit \verb|\subsection{Unterüberschrift}| begonnen wird.

\LaTeX\ bietet noch weitere Gliederungsebenen bis hin zum Unterabsatz, die aber standardmäßig nicht mehr nummeriert werden und standardmäßig auch nicht im Inhaltsverzeichnis auftauchen. Im Normalfall sind diese aber nicht unbedingt nötig.
\subsubsection{Unterunterüberschrift}
\paragraph{Absatztitel} Absatztext
\subparagraph{Unterabsatz} Absatztext

Um in der PDF-Datei einen Zeilenumbruch zu erhalten, schreiben Sie im Quelltext \verb|\\|, für einen neuen Absatz lassen Sie eine Zeile frei. Ein einzelner Zeilenumbruch im Quelltext hat keine Auswirkungen.

\section{Formatierungen, Listen, Aufzählungen}
Um die \LaTeX-Befehle für die folgenden Formatierungen zu sehen, schauen Sie sich einfach den Quellcode an.

Text kann man \textbf{fett} oder \textit{kursiv} schreiben. Sie können Text auch explizit \emph{betonen}, was dann ebenfalls kursiv dargestellt wird.

Listen sind natürlich auch möglich, diese können beliebig verschachtelt werden.
\begin{itemize}
\item Listenpunkt 1
\item Listenpunkt 2
\begin{itemize}
\item Unterpunkt
\end{itemize}
\end{itemize}

Oder eine nummerierte Aufzählung:
\begin{enumerate}
\item Listenpunkt 1
\item Listenpunkt 2
\begin{enumerate}
\item Unterpunkt
\end{enumerate}
\end{enumerate}

\section{Tabellen}
Tabellen können wie folgt erstellt werden (siehe \autoref{tab:tabelle-1}):
\begin{table}[htb]
\centering
\caption{Formatierungsparameter zur Spaltenausrichtung in Tabellen\label{tab:tabelle-1}}
\begin{tabular}{l|p{3.5cm}lcr}
	                     & \textbf{Spalte 1}                   & \textbf{Spalte 2} & \textbf{Spalte 3} & \textbf{Spalte 4} \\ \hline
	\textbf{Parameter}   & \verb|p{3.5cm}|                     & \verb|l|          &     \verb|c|      &          \verb|r| \\
	\textbf{Ausrichtung} & linksbündig mit vorgegebener Breite & linksbündig       &     zentriert     &      rechtsbündig
\end{tabular}
\end{table}

Für \qq{schönere} Tabellen empfiehlt sich das \LaTeX-Paket \verb|booktabs|, das in der Präambel mit \verb|\usepackage{booktabs}| eingebunden werden kann. Insbesondere die zugehörige Anleitung \cite{booktabs} ist lesenswert, da sie recht einfach vermittelt, was eine \qq{schöne} Tabelle ausmacht.

\section{Mathematische Gleichungen}
Gleichungen oder sonstige Formeln im Fließtext, wie z.B. $f(x)=ax+b$, können Sie einfach in Dollarzeichen \verb|$...$| einfassen. Größere Formeln können auch in einer eigenen Zeile stehen:
\begin{displaymath}
g(x) = \frac{1}{2\pi\sigma}\cdot e^{-\frac{(x-\mu)^2}{2\sigma^2}}
\end{displaymath}
Falls Sie in Ihrer Ausarbeitung viel Mathematik benötigen, lohnt sich auch ein Blick auf das \LaTeX-Paket \verb|amsmath|.
\section{Abbildungen}
Abbildungen und Diagramme können Sie direkt in einer \verb|.tex|-Datei \qq{malen}. Dazu bietet TikZ viele Möglichkeiten, eine Grafik mit Text zu beschreiben. Falls Sie nicht jeden Strich in einem Diagramm selbst definieren wollen, bietet sich die Verwendung von Pgfplots in Form der \verb|axis|-Umgebung an, wie in \autoref{fig:diagramm} gezeigt.

Falls Sie sich näher mit der Materie beschäftigen wollen, bietet sich das Manual für TikZ und PGF\cite{pgfmanual} an. Dieses Dokument ist zwar sehr umfangreich (über 1000 Seiten), bietet aber viele gut erklärte Beispiele, auch und gerade für den Einstieg.

Die folgenden Abbildungen wurden in jeweils eine \verb|figure|-Umgebung gesteckt. Diese erzeugt eine \qq{floatende} Abbildung, die von \LaTeX\ automatisch an einer \qq{geeigneten} Stelle platziert wird, also nicht zwangsläufig dort, wo sie im Quelltext definiert wurde. Wenn Sie der Abbildung eine \verb|\caption{...}| und ein \verb|\label{key}| verpassen, können Sie über dieses Label im Text per \verb|\ref{key}| oder \verb|\autoref{key}| darauf verweisen. Bei ersterem wird nur die Nummer (\qq{\ref{fig:diagramm}}), bei letzterem zusätzlich noch der Typ (\qq{\autoref{fig:diagramm}}) ausgegeben. Analog zur \verb|figure|-Umgebung wurde oben für die Tabelle die \verb|table|-Umgebung verwendet.
\begin{figure}[htb]
\centering
\begin{tikzpicture}
  \begin{axis}[xlabel=$x$,ylabel=$y$,width=0.75\textwidth]
    \addplot[smooth,mark=*,blue] plot coordinates {
        (0,2)
        (2,3)
        (3,1)
    };
    \addlegendentry{Case 1}

    \addplot[smooth,color=red,mark=x] plot coordinates {
            (0,0)
            (1,1)
            (2,1)
            (3,2)
        };
    \addlegendentry{Case 2}
  \end{axis}
\end{tikzpicture}
\caption{Diagramm aus den TikZ-Beispielen für Pgfplots\label{fig:diagramm}}
\end{figure}

Falls Sie eine Abbildung schon als separate Datei (z.B. im PDF-Format) vorliegen haben, können Sie diese mit \verb|\includegraphics{}| und dem Dateinamen einbinden, wie in \autoref{fig:smileys} geschehen. Dabei wird in den in der Präambel mit \verb|\graphicspath{...}| vorgegebenen Pfaden gesucht. Die Dateiendung wird automatisch \qq{erraten}, wenn sie nicht angegeben wird.
\begin{figure}[htb]
	\centering
	\includegraphics[width=0.75\textwidth]{smileys}
	\caption{Darstellung des Wohlbefindens anhand von Smileys\label{fig:smileys}}
\end{figure}

\cleardoublepage

\chapter{Auswertung}

\section{Metriken zur Auswertung}

% auswertung hintergrundentfernung
\section{Hintergrundentfernung}

% auswertung randentfernung
\section{Randentfernung}

% auswertung gesichtserkennung
\section{Gesichtserkennung}

\cleardoublepage

\input{chapters/diskussion.tex}
\cleardoublepage

%%% Literaturverzeichnis, lädt die Datei literatur.bib
\bibliographystyle{babplain} % "babplain" benötigt das Paket babelbib
\bibliography{literatur}
\cleardoublepage

%%% Selbständigkeitserklärung
\thispagestyle{empty}
\section*{Erklärung}
Hiermit erkläre ich, dass ich diese schriftliche Abschlussarbeit selbständig 
verfasst habe, keine anderen als die angegebenen Hilfsmittel und Quellen benutzt 
habe und alle wörtlich oder sinngemäß aus anderen Werken übernommenen Aussagen als 
solche gekennzeichnet habe.
\\[2cm]
Ort, Datum \hfil Unterschrift 

\end{document}