\documentclass[twoside,12pt,a4paper]{scrreprt}
\usepackage[T1]{fontenc}
\usepackage[utf8]{inputenc}
\usepackage[ngerman]{babel}
\usepackage{babelbib}
\usepackage{parskip}
\usepackage{microtype}
\usepackage{graphicx} % Zum Einbinden von Grafiken
\usepackage[dvipsnames]{xcolor}
\usepackage[colorlinks=true,linkcolor=Black,citecolor=MidnightBlue,urlcolor=MidnightBlue]{hyperref}
\usepackage[all]{hypcap}
\usepackage{pgfplots} \pgfplotsset{compat=1.9}
\usepackage{helvet} % Schönere SansSerif-Schrift
\usepackage{times}  % Schönere Serif-Schrift

\usepackage{blindtext} % sollte am Ende nicht mehr benötigt werden ;)

\pagestyle{headings}

\graphicspath{ {figures/} } % Pfad-Prefix für einzubindende Grafiken. Es sind auch mehrere Pfade möglich, diese müssen jeweils in eigenen {Klammern} stehen.

\setkomafont{disposition}{\normalcolor\bfseries} % überall Serifen verwenden
% oder
%\renewcommand{\familydefault}{\sfdefault} % überall Sans-Serif verwenden

% PDF-Optionen (werden in den Dateieigenschaften angezeigt)
\hypersetup{
pdftitle={Automatisches ausschneiden von Bildern aus Fotoalben},
pdfauthor={Mariella Dreißig, Tom Eckardt, Stefan Bergmann},
pdfsubject={Dokumentation Praktikum Bildverarbeitung},
pdfpagelayout=TwoColumnRight
}

%%% Eigene Makros
\newcommand{\qq}[1]{\glqq #1\grqq} % \qq{Text in Anführungszeichen}

\begin{document}

%%% Titelseite
\begin{titlepage}
\begin{center}
\LARGE Eberhard Karls Universität Tübingen\\
\large Mathematisch-Naturwissenschaftliche Fakultät \\
Wilhelm-Schickard-Institut für Informatik\\
[3cm]
\huge Dokumentation Praktikum Bildverarbeitung\\
[2cm]
\Large\textbf{Automatisches ausschneiden von Bildern aus Fotoalben}\\
[1.5cm]
\large Mariella Dreißig, Tom Eckardt, Stefan Bergmann\\
[0.5cm]
14.10.2018\\
\vfill
\small\textbf{Betreuer}\\[0.3cm]
\large Name Betreuer\\
\end{center}
\end{titlepage}

%%% Titelrückseite: Bibliographische Angaben
\thispagestyle{empty}
\vspace*{\fill}
\textbf{Mariella Dreißig, Tom Eckardt, Stefan Bergmann:}\\
\emph{Automatisches ausschneiden von Bildern aus Fotoalben}\\
Dokumentation Praktikum Bildverarbeitung \\
Eberhard Karls Universität Tübingen\\
\newpage

%%% Zusammenfassung (Abstract), hier aus externer Datei eingebunden
% !TEX root = ../ausarbeitung.tex

\begin{abstract}
\section*{Zusammenfassung}
Wissenschaftliche Arbeiten fangen normalerweise mit einer kurzen Zusammenfassung an. Deshalb sollte Ihre Arbeit ebenfalls eine solche Zusammenfassung enthalten. Die Zusammenfassung hat einen ähnlichen Inhalt wie die Motivation, nur viel kürzer. Sie soll kurz beschreiben
\begin{itemize}
\item worum es in der Arbeit geht (was war das zu lösende Problem?),
\item welche Methoden zur Problemlösung angewendet wurden,
\item wie das ganze evaluiert wurde,
\item evtl. welches Ergebnis/ Schlussfolgerungen sich daraus ergeben.
\end{itemize}

\hfil\rule{0.4\textwidth}{0.4pt}

Dieses Dokument soll als Ausgangs-Template für Bachelorarbeiten dienen. Gleichzeitig soll es zeigen, wie so ein \qq{fertiges Dokument} aussehen könnte. Um die Seiten gefüllt zu bekommen, wurde Blindtext verwendet.

Die kurzen Beschreibungen zu den Abschnitten (jeweils über dem Querstrich) wurden \cite{alexandrakirsch2016} entnommen. Diese \qq{Hinweise zum Erstellen von Bachelor-/Masterarbeiten im Arbeitsbereich Mensch-Computer-Interaktion und Künstliche Intelligenz} sind aber auch darüber hinaus zu empfehlen.
\end{abstract}
\newpage

%%% Inhaltsverzeichnis
\KOMAoption{toc}{listof,bib} % Abbildungs-/Tabellenverzeichnis, Literaturverzeichnis aufnehmen
\tableofcontents\label{toc}
\cleardoublepage

%%% Hauptteil (mit \input{dateiname} wird die Datei 'dateiname' eingebunden)
\chapter{Einleitung}

% problemstellung
% ziele definieren
% vorhandene daten beschreiben
\section{Problemstellung}
% - Ausschneiden von Fotos aus einem Fotoalbum \\
% - Fotoalbumseite sauber eingescannt \\
% - Bilder mit Rahmen, der entfernt werden soll \\
% - Gesichtserkennung mit Ausschneiden der Gesichter \\
% - Seite eines Fotoalbums besteht aus Hintergrund und eingeklebten Fotos \\
% - Einfaches modulares Programm ohne grafische Oberfläche \\
% - Erstellen von Metriken um die Ergebnisse zu überprüfen

Die hier entwickelte Software soll aus einer eingescannten Fotoalbumseite die darin enthaltenen Fotos extrahieren. Dazu sollte die Seite des Fotoalbums sauber eingescannt sein mit einem möglichst gleichfarbigen Hintergrund und ohne Spiegelungen. Die Fotos können einen Rahmen haben, der von dem Programm entfernt wird.
Sobald die Fotos ausgeschnittenen wurden kann für jedes einzelne Foto innerhalb der Fotoalbumseite eine Gesichtserkennung durchgeführt werden, bei der die erkannten Gesichter markiert und ausgeschnitten werden.
Um das Ergebnis der Software zu überprüfen soll es möglich sein die ausgeschnittenen Fotos mit Ground Truth Bildern auf Ähnlichkeit zu vergleichen. Das Programm soll über die Kommandozeile ausgeführt werden und in einzelne Module aufgeteilt sein, sodass einzelne Elemente jeder Zeit angepasst werden können.
 
\section{Benutzte Technologien}
% genutzte methoden, toolboxen, programmiersprachen
Für die Entwicklung dieser Software wurde als Programmiersprache Python 3.6 verwendet zusammen mit der Bibliothek numpy in der Version 1.14.2. Numpy ermöglicht die schnellere Berechnung von Matrizen und wird für die Bibliothek OpenCV in der Version 3.4.0.12 benötigt. OpenCV bietet eine reihe von Algorithmen für die Bildbearbeitung. Um eine Auswertung anzufertigen wurde aus der Bibliothek SciKit-image 0.13.1 die Methode zur Berechnung der strukturellen Ähnlichkeit (SSIM) verwendet.

\section{Programmaufbau}
Das Programm teilt sich in fünf Hauptkomponenten auf. Die erste ist in der \textit{main.py} zu finden, die das Programm startet und die anderen Komponenten ausführt. Die zweite Komponente kümmert sich um das entfernen des allgemeinen Hintergrunds und ist in der \textit{backgroundremover.py} zu finden. In dieser Komponente werden die eigentlichen Fotos aus dem Fotoalbum grob ausgeschnitten. Danach wird in der nächsten Komponente für jedes Foto der Rahmen entfernt, falls ein Rahmen vorhanden ist. Das entfernen des Rahmens geschieht in der \textit{rectextract.py}. Die letzte Komponente ermöglicht es Gesichter in den ausgeschnittenen Bildern zu erkenne. Dies geschieht in der \textit{facedetection.py}. Möchte man die ausgeschnittenen Bilder mit Beispielbildern vergleichen, so kann man die letzte Komponente in der Datei \textit{compare.py} verwenden. Mit deren Hilfe Metriken aufgestellt werden zum Vergleich der Bilder. In dem Flussdiagramm \ref{fig:flowchart} ist der Ablauf des Programmes nochmals genauer als Diagramm dargestellt.

\begin{figure}[h]
	\centering
	\includegraphics[width=0.45\linewidth]{images/flowchart.png}
	\caption{Flussdiagramm, das den Ablauf des Programmes darstellt. Verzweigungen können mit Hilfe von Parametern bei der Ausführung des Programmes gesteuert werden.}
	\label{fig:flowchart}
\end{figure}

\cleardoublepage

\chapter{Hintergrundentfernung}

\section{Floodfill}
Um den Hintergrund zu detektieren wird der Floodfill-Algorithmus verwendet. Dieser färbt eine Fläche zusammenhängender Pixel neu ein. Die Pixel sind zusammenhängend, wenn sich der Farbwert der Pixel nicht unterscheidet. Der Algorithmus untersucht dabei rekursiv, von einem Startpunkt beginnend, die benachbarten Pixel und färbt diese neu ein, wenn der Farbwert, der untersuchten Pixel, dem Farbwert des Startpixel entsprechen(\cite{OpenCVFloodfill}).

Der Floodfill-Algorithmus ermöglicht es eine Maske zu erstellen mit deren Hilfe der Hintergrund des Fotoalbums von den Bildern unterschieden werden kann. Um den Hintergrund zu erkennen muss dessen Farbe erkannt werden. In einem Fotoalbum ist davon auszugehen, dass die Hintergrundfarbe am häufigsten innerhalb des Bildes vorkommt. Deshalb verwenden wir die Farbe die im Histogramm des Bildes am meisten vorkommt.\\
Nachdem wir die Farbe bestimmt haben müssen wir einen Startpunkt suchen von dem aus der Floodfill-Algorithmus ausgeführt wird. Für den Startpunkt ist es nicht möglich ein willkürliches Pixel mit der Hintergrundfarbe zu wählen, da diese Farbe auch innerhalb der Bilder, die ausgeschnitten werden sollen, vorkommen kann. \\
Deshalb suchen wir nach einem Bildausschnitt der fast ausschließlich mit der Hintergrundfarbe gefüllt ist. Dazu muss beachtet werden, dass der Hintergrund nicht perfekt die gleiche Farbe hat. Aus diesem Grund verwendet das Programm die quadratische Differenz zwischen dem gesuchten Bildausschnitt und einem Ausschnitt gefüllt mit der Hintergrundfarbe. Der Bildausschnitt mit der geringsten Differenz wird dann als Startpunkt ausgewählt.\\
Um die Effizienz des Floodfill-Algorithmus zu steigern wird ein Binärbild erzeugt indem die Hintergrundfarbe weiß dargestellt wird und der gesamte Rest schwarz. Dabei wird ein Threshold angewandt um ähnliche Farben mit einem niedrigeren oder höheren Farbwert ebenfalls weiß darzustellen. \\
Auf diese Binärbild wird dann der Floodfill-Algorithmus angewandt um die zusammenhängende Fläche des Hintergrunds zu separieren, sodass die auszuschneidenden Bilder, welche die gleiche Farbe beinhalten wie der Hintergrund, nicht betroffen sind. Daraus resultiert ebenfalls ein Binärbild bei dem der Hintergrund schwarz ist und die auszuschneidenden Bilder weiß sind.  

\section{Ausschneiden}
Zur Detektion von Konturen wird das Verfahren von Satorshi Suzuki und Keiichi Abe von OpenCV verwendet, dass das Bild mit einem Rasterscan durchsucht um Kantenstartpunkte zu finden. Den Kanten wird dann gefolgt und sie werden markiert. Der Algorithmus markiert die Kanten topologisch indem er sich die vorher gefundenen Kanten merkt und so einen Baum aufspannen kann mit dem man umschlossene Kanten feststellen kann. Das Ergebnis ist eine Reihe von Konturpunkten für die Ecken der jeweiligen Kontur(\cite{Suzuki1985}).\\
Die so gefundenen Konturpunkte müssen dann überprüft werden, ob sie die kleinstmögliche Ecke für das Bild darstellen. Wenn dies geschehen ist wird das Bild an dieser Stelle mit einem kleinen Rand ausgeschnitten um Fehler in der Auswahl zu umgehen. 

\section{Validierung}
Schon während des Ausschneidens der gefundenen Bilder findet eine Validierung statt. Bei dieser Validierung werden Bilder aussortiert die entweder zu groß oder zu klein sind oder ein unnatürliches Breiten zu Höhen Verhältnis haben.
Nach dem Ausschneiden wird noch einmal separat überprüft, ob es sich um Fotos handelt indem Kanten und Ecken detektiert werden. Dies Geschieht indem man zweimal einen Gaussfilter mit unterschiedlicher Größe auf das Bild anwendet und dann die beiden Resultate voneinander abzieht. Daraus entsteht eine Approximation eines Gradientenbildes, mit dessen Hilfe man bestimmen kann, ob das Bild genug Merkmale aufweist. Ein normales Foto weißt wesentlich mehr Merkmale auf als ein Ausschnitt indem sich kein oder nur teilweise ein Foto befindet. 
\cleardoublepage

\chapter{Rahmenentfernung}

\begin{figure}[h]
 	$
 	G = \frac{1}{159}
	\begin{bmatrix}
		2 & 4 & 5 & 4 & 2 \\ 
		4 & 9 & 12 & 9 & 4 \\
		5 & 12 & 15 & 12 & 5 \\
		4 & 9 & 12 & 9 & 4 \\
		2 & 4 & 5 & 4 & 2 \\
	\end{bmatrix},	
	$
	$
	C_x = 
	\begin{bmatrix}
	-1 & 0 & +1 \\
	-2 & 0 & +2 \\
	-1 & 0 & +1
	\end{bmatrix},
	$
	$	
	C_y = 
	\begin{bmatrix}
	-1 & -2 & -1 \\
	0 & 0 & 0 \\
	+1 & +2 & +1
	\end{bmatrix},
	$
\caption{Die Figur zeigt den Gausfilterkernel sowie die Faltungsmasken, die beim Canny-Algorithmus verwendet werden}
\label{fig:gauss}
\end{figure}

Die Extraktion eines Bildes aus einem monotonen Rahmen geschieht in mehreren Schritten. Zu Beginn wird das gesamte Bild mit dem Canny-Kantendetektionsalgorithmus in ein binär Bild umgewandelt, dass sämtliche Pixel weiß darstellt bei denen eine hoher unterschied in der Intensität festzustellen ist. Dazu wird in dem Bild rauschen durch einen Gaußfilter entfernt dessen Kernel in Abbildung \ref{fig:gauss} zu erkennen ist. Um dann die Gradienten des Bildes mit zwei Faltungsmasken zu bestimmen, die auch in Abbildung \ref{fig:gauss} dargestellt sind. Die so gewonnen Gradienten werden durch eine Non-Maximum surpression überprüft, ob sie teil einer Kante sind. Daraus resultieren nur dünne Linien (\cite{OpenCVCanny}). Bei anderen Kantendetektoren wird dieser Schritt nicht ausgeführt, was im späteren dazu führen kann, dass es schwerer wird den Rahmen des Bildes zu erkennen.
 
Das Ergebnis des Canny-Kantendetektionsalgorithmus hat dann die Eigenschaft, dass stellen des Bildes mit einer monotonen Fläche nur schwarz sind und dort keine starken Gradienten durch weiße Pixel markiert wurden. Mit Hilfe dieser Eigenschaft kann man einen monotonen Rahmen um das gegebene Bild erkennen. Hierzu betrachtet man jede Seite des Bildes einzeln und betrachtet für jede Seite einen bestimmten Ausschnitt. Das erste Fenster, das betrachtet wird, hat die Höhe beziehungsweise Breite der gewählten Seite des Bildes und geht eine gegebene Anzahl von Pixeln in das Bild. Nun kann man innerhalb dieses Fensters die weißen Pixel zählen. Nachdem dies geschehen ist vergrößert man den Ausschnitt indem man weitere reihen an Pixeln des Bildes hinzunimmt und man wieder die weißen Pixel zählt. Im Schaubild \ref{fig:bordergradient} sind die Ergebnisse eines solchen Vorgehens für ein Bild mit monotonem Rahmen dargestellt. Man kann deutlich erkennen, dass die Steigung deutlich abnimmt sobald der Rahmen erreicht ist. Direkt nach dem Rahmen nimmt die Steigung wieder zu. Der Rahmen muss sich dort befinden wo der Gradient am geringsten ist und endet an dem Punkt, wo der Gradient einen Ausschlag nach oben macht. Dies kann man an den gezeichneten Gradienten in dem Schaubild \ref{fig:bordergradient} erkennen. Um danach den Rahmen zu entfernen schneidet man das Bild an dieser Stelle aus.    

\begin{figure}
	\centering
	\includegraphics[width=0.3\linewidth]{images/plot_frame/plot_top_frame.png}
	\includegraphics[width=0.3\linewidth]{images/plot_frame/plot_bottom_frame.png}\\
	\includegraphics[width=0.3\linewidth]{images/plot_frame/plot_left_frame.png}
	\includegraphics[width=0.3\linewidth]{images/plot_frame/plot_right_frame.png}
	\caption{Die Diagramme zeigen in der oberen Linie die Zunahme der weißen Pixel in den immer größer werdenden Fenstern. Außerdem zeigen sie die jeweiligen Gradienten zur vorhergehenden Anzahl an weißen Pixeln in der darunter liegenden Linie.}
	\label{fig:bordergradient}
\end{figure}    
\cleardoublepage

\chapter{Gesichtserkennung}
% 2D BV, Machine Learning -> viola-jones-methode (haar-wavelets)
% opencv Funktionsweise (Dokumentation)
% frontal, profile
% probleme, Ergebnisse
Die Aufgabenstellung beinhaltet, in den extrahierten Fotos Gesichter zu erkennen. Hier muss die Gesichtserkennung (in der Fachliteratur "Face Detection") von der Wiedererkennung von bereits identifizierten Gesichtern (in der Fachliteratur "Face Recognition") unterschieden werden. In der vorliegenden Arbeit geht es ausschließlich darum, die Position von Gesichtern in den eingescannten Fotos zu erkennen und nicht, bereits erkannte Gesichter wiederzuerkennen oder zuzuordnen \footnote{Aufgrund der geringen Datenmenge und der schlechten Qualität der alten Fotos ist es nicht möglich, Gesichter wiederzuerkennen.}.\\
In der klassischen Bildverarbeitung wurden lange einfache 2D-Verfahren verwendet, um Gesichter in Bildern zu erkennen (\cite{Kees2012}, \cite{Wachter2001}). Beim Template Matching werden beispielsweise Vorlagen (von Teilen) von Gesichtern erzeugt um diese dann im Anwendungsfall mit den Zielbildern zu vergleichen. Dieses Verfahren ist nicht sehr robust, besonders bei sich verändernden Lichtverhältnissen, aber gleichzeitig extrem rechenaufwändig. Eine einfachere Methode der Gesichtserkennung in der klassischen Bildverarbeitung ist die Nutzung von geometrischen Merkmalen in den Gesichtern. Hierbei werden beispielsweise Grauwerte in einem bestimmten Bereich des Bildes aufsummiert oder Gradientenoperatoren zur horizontalen und vertikalen Kantenerkennung angewandt. So kann die Ausrichtung, die relative Lage und ähnliche Merkmale der Nase, der Augen und des Munden identifiziert und zusammen als Gesicht klassifiziert werden. Doch auch diese Methode ist rechen- und zeitintensiv. Deswegen wird häufig eher auf komplexere Methoden zur Gesichtserkennung gesetzt, wie beispielsweise das Elastic Bunch Graph Matching, das eine robuste Waveletanalyse nutzt (\cite{Wiskott1999}) oder die auf der Hauptkomponentenanalyse basierende Eigenfaces-Methode (\cite{doi:10.1162/jocn.1991.3.1.71}). Weiterentwicklungen dieser beiden Methoden werden heute noch eingesetzt, obwohl die Ansätze schon vergleichsweise alt sind.
% https://en.wikipedia.org/wiki/Viola%E2%80%93Jones_object_detection_framework 
% https://de.wikipedia.org/wiki/Viola-Jones-Methode

% inspiration 
	% https://www.heise.de/select/ix/2017/11/1509393829637286
	% https://eliteinformatiker.de/2013/08/28/eine-einfache-gesichtserkennung-mit-opencv-und-scikit-learn
	% https://realpython.com/face-recognition-with-python/
	% https://www.superdatascience.com/opencv-face-detection/
% opencv doku https://docs.opencv.org/3.4/d7/d8b/tutorial_py_face_detection.html
\cleardoublepage

\chapter{Auswertung}

\section{Metriken zur Auswertung}

% auswertung hintergrundentfernung
\section{Hintergrundentfernung}

% auswertung randentfernung
\section{Randentfernung}

% auswertung gesichtserkennung
\section{Gesichtserkennung}

\cleardoublepage

\chapter{Benutzung und Konfiguration}

\section{Ausführen des Programmes}

Für die Installation ist Python 3.6 notwendig mit dem Paketmanager Pip. Um die Abhängigkeiten zu installieren muss man mit der Kommandozeile in das Hauptverzeichnis navigieren indem sich die Datei \textit{requirements.txt} befindet. Dort führt man folgenden Befehl aus: 
\begin{lstlisting}
pip3 install -r requirements.txt
\end{lstlisting}
Pip installiert nun sämtliche Abhängigkeiten, die zur Ausführung des Programmes benötigt werden. Um das Programm dann auszuführen genügt folgender Befehl:
\begin{lstlisting}
python3 main.py /path/to/photo/album/page.tif
\end{lstlisting}
Man kann dann mit Parametern steuern, wo die ausgeschnittenen Fotos gespeichert werden sollen, wo sich die Konfigurationsdatei befindet, ob man Gesichter in den Fotos detektiert haben möchte oder ob man ein Bild mit einem Groud Truth Bild vergleichen möchte. Die Anleitung zum Ausführen dieser Befehle erhält man mit folgendem Befehl:
\newpage
\begin{lstlisting}
python3 main.py -h 
  usage: main.py [-h] [--result RESULT] [--compare COMPARE] 
                 [--config CONFIG] [--face] image

  positional arguments:
    image              path to the image of the photo album page

  optional arguments:
    -h, --help         show this help message and exit
    --result RESULT    path to the directory to store the 
                       results
    --compare COMPARE  path to the image to compare 
                       the given image to
    --config CONFIG    path to the config file
    --face             enables face detection
\end{lstlisting}  

\section{Konfigurationsdatei und Parameter}
In der Konfigurationsdatei \textit{config} befinden sich Parameter mit denen man das Programm steuern kann. Sämtliche Parameter die man setzen kann sind im folgenden aufgeführt mit einer Beschreibung was mit diesem Parameter gesteuert wird.

%or \small or \footnotesize etc.

\begin{lstlisting}
#----------------------------
# Parameter for removing the background
#----------------------------
[BackgroundRemover]
# The size of the searched spot in the background of the images
# to start the flood fill algorithm. The higher the value, 
# the slower the search but up to a certain size it is more stable.
SpotSize = 200

# The threshold around the primary background color, 
# which still should belong to the background to create a
# binary image which is separating the background 
# colors from the rest.
BinaryThreshold = 25

# The size an image must at least have to be considered as an 
# image.
MinImageSize = -100

# The maximum relation between the width and the height of an 
# image to prevent the finding of strips.
MaxRelationImageDimensions = 2.5

# The padding with which image is cut out of the original photo.
ImagePadding = 10

# The necessary amount of features needed to be regarded as image. 
# To check if the found image is legit.
FeatureThreshold = 10
\end{lstlisting}
\begin{lstlisting}
#----------------------------
# Parameter for extracting the image
#----------------------------
[ImageExtraction]
# The maximum size of the window within the program searches
# a frame around the image.
MaxWindowSize = 0.1

# The number of steps to search for the frame. 
# The maximum window size and the number of steps determine 
# the size of the steps fitting to the image size.
Steps = 25

# The ignored steps. This is done to prevent the program 
# of finding the already given image boarder.
Offset = 4
\end{lstlisting}
\newpage	
\begin{lstlisting}
#----------------------------
# Parameter for the face detection
#----------------------------
[FaceDetection]
# path for the Haarcascade files for frontal faces
CascadePathFrontal = ./imextract/cascade/haarcascade_frontalface.xml

# path for the Haarcascade files for profile faces
CascadePathProfile = ./imextract/cascade/haarcascade_profileface.xml

# specifies how much the image size is reduced at each image scale
ScaleFactor = 1.2

# specifies how many neighbors each candidate rectangle 
# should have to retain it
Neighbors = 5
\end{lstlisting}

\cleardoublepage

%%% Literaturverzeichnis, lädt die Datei literatur.bib
\bibliographystyle{babplain} % "babplain" benötigt das Paket babelbib
\bibliography{literatur}
\cleardoublepage

\end{document}