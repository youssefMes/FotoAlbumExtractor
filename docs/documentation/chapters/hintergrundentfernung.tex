\chapter{Hintergrundentfernung}

\section{Floodfill}
Um den Hintergrund zu detektieren wird der Floodfill-Algorithmus verwendet. Dieser färbt eine Fläche zusammenhängender Pixel neu ein. Die Pixel sind zusammenhängend, wenn sich der Farbwert der Pixel nicht unterscheidet. Der Algorithmus untersucht dabei rekursiv, von einem Startpunkt beginnend, die benachbarten Pixel und färbt diese neu ein, wenn der Farbwert, der untersuchten Pixel, dem Farbwert des Startpixel entsprechen(\cite{OpenCVFloodfill}).

Der Floodfill-Algorithmus ermöglicht es eine Maske zu erstellen mit deren Hilfe der Hintergrund des Fotoalbums von den Bildern unterschieden werden kann. Um den Hintergrund zu erkennen muss dessen Farbe erkannt werden. In einem Fotoalbum ist davon auszugehen, dass die Hintergrundfarbe am häufigsten innerhalb des Bildes vorkommt. Deshalb verwenden wir die Farbe die im Histogramm des Bildes am meisten vorkommt.\\
Nachdem wir die Farbe bestimmt haben müssen wir einen Startpunkt suchen von dem aus der Floodfill-Algorithmus ausgeführt wird. Für den Startpunkt ist es nicht möglich ein willkürliches Pixel mit der Hintergrundfarbe zu wählen, da diese Farbe auch innerhalb der Bilder, die ausgeschnitten werden sollen, vorkommen kann. \\
Deshalb suchen wir nach einem Bildausschnitt der fast ausschließlich mit der Hintergrundfarbe gefüllt ist. Dazu muss beachtet werden, dass der Hintergrund nicht perfekt die gleiche Farbe hat. Aus diesem Grund verwendet das Programm die quadratische Differenz zwischen dem gesuchten Bildausschnitt und einem Ausschnitt gefüllt mit der Hintergrundfarbe. Der Bildausschnitt mit der geringsten Differenz wird dann als Startpunkt ausgewählt.\\
Um die Effizienz des Floodfill-Algorithmus zu steigern wird ein Binärbild erzeugt indem die Hintergrundfarbe weiß dargestellt wird und der gesamte Rest schwarz. Dabei wird ein Threshold angewandt um ähnliche Farben mit einem niedrigeren oder höheren Farbwert ebenfalls weiß darzustellen. \\
Auf diese Binärbild wird dann der Floodfill-Algorithmus angewandt um die zusammenhängende Fläche des Hintergrunds zu separieren, sodass die auszuschneidenden Bilder, welche die gleiche Farbe beinhalten wie der Hintergrund, nicht betroffen sind. Daraus resultiert ebenfalls ein Binärbild bei dem der Hintergrund schwarz ist und die auszuschneidenden Bilder weiß sind.  

\section{Ausschneiden}
Zur Detektion von Konturen wird das Verfahren von Satorshi Suzuki und Keiichi Abe von OpenCV verwendet, dass das Bild mit einem Rasterscan durchsucht um Kantenstartpunkte zu finden. Den Kanten wird dann gefolgt und sie werden markiert. Der Algorithmus markiert die Kanten topologisch indem er sich die vorher gefundenen Kanten merkt und so einen Baum aufspannen kann mit dem man umschlossene Kanten feststellen kann. Das Ergebnis ist eine Reihe von Konturpunkten für die Ecken der jeweiligen Kontur(\cite{Suzuki1985}).\\
Die so gefundenen Konturpunkte müssen dann überprüft werden, ob sie die kleinstmögliche Ecke für das Bild darstellen. Wenn dies geschehen ist wird das Bild an dieser Stelle mit einem kleinen Rand ausgeschnitten um Fehler in der Auswahl zu umgehen. 

\section{Validierung}
Schon während des Ausschneidens der gefundenen Bilder findet eine Validierung statt. Bei dieser Validierung werden Bilder aussortiert die entweder zu groß oder zu klein sind oder ein unnatürliches Breiten zu Höhen Verhältnis haben.
Nach dem Ausschneiden wird noch einmal separat überprüft, ob es sich um Fotos handelt indem Kanten und Ecken detektiert werden. Dies Geschieht indem man zweimal einen Gaussfilter mit unterschiedlicher Größe auf das Bild anwendet und dann die beiden Resultate voneinander abzieht. Daraus entsteht eine Approximation eines Gradientenbildes, mit dessen Hilfe man bestimmen kann, ob das Bild genug Merkmale aufweist. Ein normales Foto weißt wesentlich mehr Merkmale auf als ein Ausschnitt indem sich kein oder nur teilweise ein Foto befindet. 